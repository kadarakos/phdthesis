\section{Baseline Results}\label{sec:results}

The experiments presented here set the baseline performance 
for the visually grounded bilingual models
and introduces the data settings that we will use in the 
later sections.

\paragraph{Aligned}
%\todo[inline]{Results of training the model on only the M30K}

\begin{table}[t]
    \centering
    \renewcommand{\arraystretch}{1.0}
    \begin{tabular}{lccc}
        \toprule
         & \multicolumn{3}{c}{English}\\
         \cmidrule(lr){2-4}
         & I $\rightarrow$ T & T $\rightarrow$ I & Sum\\
         \midrule
         En & 40.5 & 28.8 & 346.4 \\
         \; + De & 41.4 & 29.9  & 352.8\\
         \; \; + c2c & 42.8  & 32.1 & 361.6\\
         COCO & 34.4 & 24.8 & 304.0 \\
         \; + En & \textbf{46.2} & \textbf{33.4} & \textbf{374.4}\\
         \bottomrule
    \end{tabular}
    %\caption{Recall @ 1 and Sum-of-Recall-Scores for Image-to-Text (I$\rightarrow$T) and 
    %Text-to-Image (T$\rightarrow$I) on the English
    %M30K 2016 test set in the \emph{aligned setting}: 
    %only the M30K English (En), M30K German and English (+De), 
    %and with the additional caption ranking objective (+c2c).}
    \caption{Performance on the English
    M30K 2016 test set in the \emph{aligned setting} for models trained on 
    M30K English (En), both M30K German and English (+De), 
    and with caption ranking (+c2c).}
    \label{tab:engbaseline}
\end{table}



\begin{table}
    \centering
    \renewcommand{\arraystretch}{1.0}   
    \begin{tabular}{lccc}
        \toprule
         & \multicolumn{3}{c}{German} \\
         \cmidrule(lr){2-4}
         & I $\rightarrow$ T & T $\rightarrow$ I & Sum\\
         \midrule
         De & 34.9 & 24.6 & 311.2 \\
         \; + En & \textbf{38.6} & 26.0 & 324.6\\
         \; \; + c2c & 38.3 & \textbf{27.7} &   \textbf{334.0} \\
         \; + COCO & 36.4 & 25.7 & 319.7\\
         \bottomrule
    \end{tabular}
        \caption{Performance on the German
    M30K 2016 test set in the \emph{aligned setting} for models trained on 
    M30K German (De), both M30K German and English (+En), 
    and with caption ranking (+c2c).}
    \label{tab:gerbaseline}
\end{table}

\begin{table*}[h!]
    \centering
    \renewcommand{\arraystretch}{1.0}
    \begin{tabular}{lcccccc}
        \toprule
         & \multicolumn{3}{c}{English} & \multicolumn{3}{c}{German}\\
        \cmidrule(lr){2-4} \cmidrule(lr){5-7}
         & I $\rightarrow$ T & T $\rightarrow$ I & Sum & I $\rightarrow$ T & T $\rightarrow$ I & Sum \\
         \midrule
         En+De+c2c & 42.8 & 28.6 & 361.6 & 38.3 & 27.7 & 334.0 \\
         \; + COCO & \textbf{46.5} & \textbf{34.8} &  \textbf{378.9}  & \textbf{40.6} & \textbf{28.8} & \textbf{344.6}\\
         \bottomrule
    \end{tabular}
    \caption{Recall @ 1 and Sum-of-Recall-Scores for Image-to-Text (I $\rightarrow$ T) and Text-to-Image (T $\rightarrow$ I) baseline results on the English and German
    M30K 2016 test in the \emph{aligned plus disjoint} setting}\label{tab:alignedplus}
\end{table*}

In these experiments we only use the \emph{aligned} English-German data from M30K. 
%This is the setting was studied before in \newcite{D17-1303} and \newcite{kadar2018conll}.
Tables~\ref{tab:engbaseline} and \ref{tab:gerbaseline} present the result for English and German, respectively. The Sum-of-recall scores for both languages show that the best approach is the bilingual model with the c2c loss (En+De+c2c, and De+En+c2c). These results reproduce the findings of
\cite{kadar2018conll}.


\paragraph{Disjoint}\label{sec:baseline_results:disjoint}
%\todo[inline]{Results of training the model on only the M30K}

\begin{comment}
\begin{table}
    \centering
    \renewcommand{\arraystretch}{1.0}
    \begin{tabular}{lccc}
        \toprule
         & \multicolumn{3}{c}{English}\\
         & I $\rightarrow$ T & T $\rightarrow$ I & Sum\\
         \midrule
          COCO & 34.4 & 24.8 & 304.0 \\
         En & 40.5 & 28.8 & 346.4 \\
         \; + COCO & \textbf{46.2} & \textbf{33.4} & \textbf{374.4}\\
         \bottomrule
    \end{tabular}
     \caption{Measuring \emph{domain shift} by testing models trained on
     COCO (COCO), M30K English (En) and both (+COCO) on
     the English M30K 2016 test set.}
    \label{tab:engdisjoint}
\end{table}
\end{comment}

%\begin{table}[t]
%    \centering
%    \renewcommand{\arraystretch}{1.0}
%    \begin{tabular}{lccc}
%        \toprule
%         & \multicolumn{3}{c}{German}\\
%         & I $\rightarrow$ T & T $\rightarrow$ I & Sum\\
%         \midrule
%         De & 34.9 & 24.6 & 311.2 \\
%         \; + COCO & \textbf{36.4} & \textbf{25.7} & \textbf{319.7}\\
%         \bottomrule
%    \end{tabular}
%    \caption{Improving the performance of the German (De) model trained on the M30K German set in the \emph{disjoint} setting by adding COCO (+COCO). Results reported on the German M30K 2016 test set.}\label{tab:gerdisjoint}
%\end{table}


We now determine the performance of the model when it is trained on data drawn from different data sets with no overlapping images.

First we train two English \emph{monolingual}
models: one on the M30K English dataset and one on the English COCO dataset. Both models are evaluated on image--sentence ranking performance on the M30K English test 2016 set. The results in Table~\ref{tab:engbaseline} show that there is a substantial difference in performance in both text-to-image and image-to-text retrieval, depending on whether the model is trained on the M30K or the COCO dataset.
The final row of Table~\ref{tab:engbaseline} shows, however, 
that jointly training on both data sets improves over 
only using the M30K English training data.

We also conduct experiments in the \emph{bilingual disjoint} setting, 
where we study whether it is possible to improve the performance of a German model using the
out-of-domain English COCO data. Table~\ref{tab:gerbaseline}
shows that there is an increase in performance when the model is trained on the disjoint sets, as opposed to
only the in-domain M30K German (compare De against De+COCO). This result is not too surprising as we have observed both the advantage of joint training on both languages in the \emph{aligned} setting and the overlap between the different datasets. 

Finally, we compare the performance of a German model trained in the \emph{aligned}
and \emph{disjoint} settings. We find that a model trained in the \emph{aligned}
setting (De+En) is better than a model trained in the \emph{disjoint} setting (De+COCO), as shown in Table \ref{tab:gerbaseline}.
This finding contradicts the conclusion of \citet{kadar2018conll}, who claimed that the
\emph{aligned} and \emph{disjoint} conditions lead to comparable performance. This is most likely because the disjoint setting in \cite{kadar2018conll} is artificial, in the sense that they used different 50\% subsets of M30K. In our experiments the disjoint image--caption sets are real, in the sense that we trained the models on the two different datasets.
 
%Comparing line 1. and
%line 2, in Table~\ref{tab:engbaseline} we observe that training on COCO results in much lower
%performance then training on M30K, showing a considerable difference in domains. However, 
%line 3. shows that there is a considerable overlap too and jointly training on COCO and M30K 
%results in much higher performance than only training on English M30K. This result is reflected
%in line 2. in Table~\ref{tab:gerdisjoint}, where we see that training jointly on COCO and 
%the German M30K improves over the German M30K monolingual model. However, comparing the 
%results between line 2. in Table~\ref{tab:gerbaseline} and line 2. in Table~\ref{tab:gerdisjoint}
%shows that training on \emph{aligned} data is much more beneficial than on \emph{disjoint}.

\paragraph{Aligned plus disjoint}
\begin{comment}

\begin{table}[t]
    \centering
    \renewcommand{\arraystretch}{1.0}
    \begin{tabular}{lccc}
        \toprule
         & \multicolumn{3}{c}{English}\\
         & I $\rightarrow$ T & T $\rightarrow$ I & Sum\\
         \midrule
         En+De+c2c & 42.8 & 28.6 & 361.6 \\
         \; + COCO & \textbf{46.5} & \textbf{34.8} &  \textbf{378.9}\\
         \bottomrule
    \end{tabular}
    \caption{Recall @ 1 and Sum-of-Recall-Scores for Image-to-Text (I$\rightarrow$T) and 
    Text-to-Image (T$\rightarrow$I) baseline results on the English
    M30K 2016 test set for models in the \emph{aligned plus disjoint} setting: 
    M30K German-English model with c2c loss  ( En+De+c2c) and the same model with CO (+COCO)}\label{tab:engalignedplus}
\end{table}


\begin{table}[ht]
    \centering
    \renewcommand{\arraystretch}{1.0}
    \begin{tabular}{lccc}
        \toprule
         & \multicolumn{3}{c}{English}\\
         & I $\rightarrow$ T & T $\rightarrow$ I & Sum\\
         \midrule
        En+De+c2c & 38.3 & 27.7 & 334.0 \\
         \; + COCO & \textbf{40.6} & \textbf{28.8} & \textbf{344.6} \\
             \bottomrule
    \end{tabular}
    \caption{Recall @ 1 Image-to-Text and Text-to-Image baseline results  on the German
    M30K 2016 test set.}\label{tab:geralignedplus}
\end{table}
\end{comment}


Our final baseline experiments explore the combination of 
\emph{disjoint} and \emph{aligned} data settings. We train an English-German bilingual model with the c2c
objective on M30K, and we also train on the English COCO data. Table~\ref{tab:alignedplus} shows that adding the 
\emph{disjoint} data improves performance for
both English and German compared to training 
solely the \emph{aligned} model. 

\paragraph{Summary}
First we reproduced the findings of \cite{kadar2018conll} 
showing that bilingual joint
training improves over monolingual and using c2c loss
further improves performance. 
Furthermore, we have found that adding the COCO as 
additional training data both when only training on German, and training on both German-English from 
M30K improves performance 
even if the model is trained on data drawn from a different dataset.




%\subsection{Domain shift}

%\todo[inline]{Results of training of model on only COCO and evaluating on M30K.}