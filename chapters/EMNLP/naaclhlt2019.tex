%
% File naacl2019.tex
%
%% Based on the style files for ACL 2018 and NAACL 2018, which were
%% Based on the style files for ACL-2015, with some improvements
%%  taken from the NAACL-2016 style
%% Based on the style files for ACL-2014, which were, in turn,
%% based on ACL-2013, ACL-2012, ACL-2011, ACL-2010, ACL-IJCNLP-2009,
%% EACL-2009, IJCNLP-2008...
%% Based on the style files for EACL 2006 by 
%%e.agirre@ehu.es or Sergi.Balari@uab.es
%% and that of ACL 08 by Joakim Nivre and Noah Smith

% THE LINK https://www.overleaf.com/3661239131qwycrvwrdbsx

%\documentclass[11pt,a4paper]{article}
%\usepackage[draft]{hyperref}
%\usepackage{naaclhlt2019}
%\usepackage{times}
%\usepackage{latexsym}
%\usepackage[utf8]{inputenc}
%\usepackage{tikz}
%\def\checkmark{\tikz\fill[scale=0.4](0,.35) -- (.25,0) -- (1,.7) -- (.25,.15) -- cycle;} 
 %\usepackage{algpseudocode}
 %\usepackage{algorithm}
%\usepackage{amssymb}
 %\usepackage{comment}
 %\usepackage{booktabs}
 %\usepackage{amsmath}
 %\usepackage{arydshln}
% \usepackage{subcaption}
 %\usepackage{amsfonts}
 %\usepackage{subfig}
 %\usepackage{float}





%\usepackage{url}
%\usepackage{todonotes}
%\aclfinalcopy % Uncomment this line for the final submission
%\def\aclpaperid{***} %  Enter the acl Paper ID here

%\setlength\titlebox{5cm}
% You can expand the titlebox if you need extra space
% to show all the authors. Please do not make the titlebox
% smaller than 5cm (the original size); we will check this
% in the camera-ready version and ask you to change it back.

%\newcommand\BibTeX{B{\sc ib}\TeX}

\chapter{Synthetic Pairs for Multilingual Grounded Language Learning}
\label{ch:EMNLP}

\paragraph{abstract}

Recent work has highlighted the advantage of jointly learning grounded sentence representations 
from multiple languages. 
However, the data used in these studies has 
been limited to an \emph{aligned} scenario, in which
the same images are annotated with sentences 
in multiple languages. In this paper, we focus on the more realistic 
\emph{disjoint} scenario using English and German
image--caption data sets where the images
do not overlap between languages. 
We observe that training with aligned data provides larger gains than training with disjoint data, which may be caused by the lack of coherence between the disjoint data sets. 
%\todo{The problem that we solve is not cross-domain is just the lack
%of alignment.}
To address the lack of coherence, we propose a
\emph{pseudopairing} method, in which we create \emph{synthetically aligned} 
English--German--Image triplets from the disjoint sets. The pseudopairs are created in a two-step process:
first we train a model on the 
\emph{disjoint} data, then we create novel pairs across data sets using sentence similarity 
under the learned model.
Experiments show that the pseudopair method improves 
image--sentence retrieval performance, 
%both when \emph{aligned} and \emph{disjoint} sets are available, 
%or when there are only \emph{disjoint} sets, 
despite requiring no external training data or models. 
We do find, however, that using an external machine translation model to generate the synthetic data sets results
in better performance.

 %\captionsetup[subfigure]{labelformat=empty}

%\author{First Author \\
 % Affiliation / Address line 1 \\
 % Affiliation / Address line 2 \\
 % Affiliation / Address line 3 \\
 % {\tt email@domain} \\\And
 % Second Author \\
 % Affiliation / Address line 1 \\
  %Affiliation / Address line 2 \\
  %Affiliation / Address line 3 \\
  %{\tt email@domain} \\}

%\date{}



%\begin{abstract}

Recent work has highlighted the advantage of jointly learning grounded sentence representations 
from multiple languages. 
However, the data used in these studies has 
been limited to an \emph{aligned} scenario, in which
the same images are annotated with sentences 
in multiple languages. In this paper, we focus on the more realistic 
\emph{disjoint} scenario using English and German
image--caption data sets where the images
do not overlap between languages. 
We observe that training with aligned data provides larger gains than training with disjoint data, which may be caused by the lack of coherence between the disjoint data sets. 
%\todo{The problem that we solve is not cross-domain is just the lack
%of alignment.}
To address the lack of coherence, we propose a
\emph{pseudopairing} method, in which we create \emph{synthetically aligned} 
English--German--Image triplets from the disjoint sets. The pseudopairs are created in a two-step process:
first we train a model on the 
\emph{disjoint} data, then we create novel pairs across data sets using sentence similarity 
under the learned model.
Experiments show that the pseudopair method improves 
image--sentence retrieval performance, 
%both when \emph{aligned} and \emph{disjoint} sets are available, 
%or when there are only \emph{disjoint} sets, 
despite requiring no external training data or models. 
We do find, however, that using an external machine translation model to generate the synthetic data sets results
in better performance.
\end{abstract}
 
 
 %\begin{comment}
 
 %Recent work on learning grounded sentence representations has
 %been explored in the multilingual setting. 
% However, the data in these studies has been limited to a setup where
% the same images  $\mathcal{I}$ are annotated with sentences in 
% multiple languages:
% $\mathcal{I}$-$\mathcal{L}_1$-$\mathcal{L}_2$. We are interested
% in improving these representations using out-of-domain data as it is
%monolingual \emph{disjoint} datasets are more likely to be found:
% $\mathcal{I}_1$-$\mathcal{L}_1$ and $\mathcal{I}_2$-$\mathcal{L}_2$. 
% In our experiments we find that training on out-of-domain data is 
% only useful when a model is co-trained with in-domain data. 
% To improve performance in this \emph{disjoint} setting we automatically extend these disjoint datasets into and generated \emph{aligned} datasets in order to benefit from a cross-lingual caption--caption ranking objective, as well as the image--caption ranking objective. We introduce a method for generating $\mathcal{I}_2$-$\hat{\mathcal{L}_1}$ pseudo-pairs using a pre-trained $\mathcal{I}$-$\mathcal{L}_1$-$\mathcal{L}_2$ model. Given an $\mathcal{I}_2$-$\mathcal{L}_2$ dataset, our method creates the pseudo-pairs by transferring the most similar $\mathcal{L}_1$ sentences from the $\mathcal{I}$-$\mathcal{L}_1$-$\mathcal{L}_2$ training data. 
% We evaluate our approach by creating Image-German pseudopairs for the English MS COCO dataset, using a model trained on the Multi30K English-German dataset. We find that our introduced pseudopair method 
% improves performance even though it does not require us to introduces any new data or models. Nevertheless, we do find that our approach falls short of using a good quality pre-trained translation system 
% to automatically create annotations in the other language.
% \end{comment}
 
% \begin{abstract}
%Previous work have looked at learning multilingual visually grounded sentence embeddings and have shown the benefit of multilingual joint training. However, in these works the same images were annotated with sentences from multiple languages. We argue that a more realistic scenario is when such an alignment is not available. Here we explore how the multilingual visually grounded sentence embedding methods perform in such a disjoint setting. Furthermore, we assess the possibility if improving the results in this setup with distant supervision 1.) by using off-the-shelf machine translation to obtain more bi-lingual data, 2.) we introduce a method to generate new image-sentence pseudo-pairs between data sets without the use of extrnal models or data sets.
%\end{abstract}

\section{Introduction}

The perceptual-motor system plays an important role 
in concept acquisition and representation, 
and in learning the meaning of linguistic expressions
\citep{pulvermuller2005brain}. In natural language 
processing, many approaches have been proposed that
integrate visual information in the learning of 
word and sentence representations, highlighting
the benefits of visually grounded representations
\citep{lazaridou2015combining,baroni2016grounding,kiela2017learning,elliott2017imagination}.
In these approaches the visual world is
taken as a naturally occurring meaning representation for 
linguistic utterances, grounding language in perceptual
reality.

 Recent work has shown that we can learn better 
 visually grounded representations for sentences by
 jointly training image--sentence ranking models on 
 multiple languages \citep{gella2017image,R17-1020,kadar2018conll}. 
 %The main
 %advantage of multilingual training is that the model
 %can be jointly optimized for image--sentence ranking and sentence--sentence ranking across languages.
  This line of research has
 focused on training models on datasets where the same 
 images are annotated with sentences in multiple languages. 
 This \emph{alignment} has
 either been in the form of the translation pairs 
 (e.g. the German, English, French, and Czech data in
 Multi30K \citep{W16-3210}) or independently collected
 sentences (English and Japanese captions in STAIR
 \citep{Yoshikawa2017}). 
 %\todo[inline]{Has anyone actually trained a model for image--sentence retrieval on STAIR Captions? If not, we can expect that training an En-Jp model would work similarly to training an En-De model on the Multi30K.}
 
 In this paper, we consider the problem of training an image--sentence ranking model using image-caption collections 
 in different languages with non-overlapping images drawn from different sources. 
 We call these collections \emph{disjoint} datasets, as opposed to
 \emph{aligned} datasets. \cite{kadar2018conll} showed that a
 model trained on disjoint datasets performs on-par with a model
 trained on aligned data. However, the disjoint datasets in their paper are artificial because they were formed by randomly splitting the Multi30K dataset into two halves. We examine whether the ranking model
 can benefit from multilingual supervision when it is
 trained using disjoint datasets drawn from different
 sources. In experiments with the Multi30K and COCO datasets,
 we find substantial benefits from training with these disjoint sources, but the best performance 
 comes from training on aligned datasets.
 
 %Furthermore, the model with overlapping data can be improved with an additional caption--caption retrieval objective, which is not available when a model is trained with disjoint data. 
 %and further improvements are possible by sentence--sentence ranking \cite{D17-1303,kadar2018conll}. 
 
 Given the empirical benefits of training on aligned datasets, 
 we explore two approaches to creating synthetically 
 aligned training data in the \emph{disjoint} scenario. 
 One approach to creating synthetically aligned data is to use an
 off-the-shelf machine translation system to generate 
 new image-caption pairs by translating the original
 captions.
 This approach is very simple, but has the limitation 
 that an external system needs to be trained, which requires additional data. 
 
 The second approach is to generate synthetically aligned data that are \emph{pseudopairs}. We assume the existence of image--caption datasets in different languages where the images do not overlap between the datasets. Pseudopairs are created by annotating the images of one dataset with the captions from another dataset. This can be achieved by leveraging the sentence similarities predicted by an image-sentence ranking model trained on the original image--caption datasets. 
 One advantage of this approach is that does not
 require additional models or datasets because it uses the 
 trained model to create new pairs. 
 The resulting pseudopairs can then be used 
 to re-train or fine-tune the original model. 
%given two image--caption datasets data sets $D_1$ and $D_2$, 
%consisting of image-caption pairs $<i_1, c_1>$ and $<i_2, c_2>$, we create new pairs either in the direction $D_1 \rightarrow D_2$ leading 
%to pairs $<i_2, c_1>$, or vice-versa $D_2 \rightarrow D_1$ 
%yielding pairs of $<i_1, c_2>$. 

In experiments on the Multi30K and COCO datasets, we find that the pseudopair approach consistently improves 
performance compared to training on the disjoint datasets, but the improvements less than using the translation approach. Nevertheless, the results show the potential for model-based data synthesis to improve the learning of visually grounded sentence representations. %\todo[inline]{Draw some kind of early conclusion for the reader here? What are the implications of these findings?}

%We hypothesise that \emph{sufficiently high-quality} pseudo-pairs can improve the quality of a multilingual image--sentence ranking model that is trained over disjoint datasets.


%as noisy data to train the caption--caption ranking model, and caption$_i$--image$_j$ and caption$_j$--image$_i$ ranking models. We expect that the additional training data extracted from the unaligned datasets will improve image--sentence ranking performance compared to training monolingual models.
%In this paper, we use distant supervision to train CNN-RNN multilingual image--sentence ranking models with unaligned datasets.



%Given  D$_i$ and D$_j$, our approach is to find pairs caption$_i$--image$_i$ --- caption$_j$--image$_j$ in the unaligned datasets using the learned image representation model. 


\section{Method}

%\begin{figure*}[th!]
%  \includegraphics[width=1\textwidth]{assets/overview}
%  \caption{An overview of our approaches to creating an aligned dataset $\mathcal{D}_2$ of images ($\mathcal{I}$) with captions in two languages $\mathcal{C}^{\ell_1}$ and $\mathcal{C}^{\ell_2}$, from a $\mathcal{D}_2$ that only consists of $\mathcal{I}$-$\mathcal{C}^{\ell_2}$ pairs. In (a), we use an off-the-shelf machine translation system to create the $\hat{\mathcal{L}_2}$ data. The $\mathcal{D}_1$ and now-completed $\mathcal{D}_2$ data is used to train an image--sentence ranking model. In (b), we first train an ranking model on the $\mathcal{D}_1$ data, then we use the resulting Model$_0$ to transfer $\hat{\mathcal{L}_2}$ annotations given the $\mathcal{L}_1$ data to create pseudopairs. The resulting $\mathcal{D}_1$ and now-completed $\mathcal{D}_2$ data is used to train the final image--sentence ranking model. In (c), we follow the same process as (b), except that the initial model is fine-tuned with the pseudopairs.}\label{fig:overview}
%\end{figure*}

We adopt the model architecture and training procedure
of \citet{kadar2018conll} for the task of matching images with sentences. This task is defined as learning to rank the sentences associated with an image higher than other sentences in the data set, and vice-versa \cite{hodosh2013framing}. The model is comprised of 
a recurrent neural network language model and a 
convolutional neural network image encoder. 
The parameters of the 
language encoder are randomly initialized, while
the image encoder pre-trained, frozen during 
training and followed by a linear layer which is
tuned for the task.  
The model is trained to make true pairs $<a,b>$ 
similar to each other, and contrastive pairs
$<\hat{a},b>$ and 
$<a,\hat{b}>$ dissimilar from each other in a joint
embedding space by minimizing the max-violation loss function
\cite{faghri2017vse++}:
%
\begin{equation}
\small{
\label{eq:maxviol}
\begin{split}
\mathcal{J}(a, b) = &\max_{<\hat{a}, b>}[\text{max}(0, \alpha - s(a,b) + s(\hat{a}, b))] \;+ \\ &\max_{<a, \hat{b}>}[\text{max}(0, \alpha - s(a,b) + s(a, \hat{b}))]
\end{split}
}
\end{equation}

In our experiments, the $<a, b>$ pairs are either image-caption pairs 
$<i, c>$ or caption--caption pairs $<c_a, c_b>$ (following \cite{D17-1303,kadar2018conll}).
When we train on $<i, c>$ pairs, we sample a batch
from an image--caption data set with uniform probability, encode the images and the sentences, and perform an update of the model parameters.
For the caption--caption objective, 
we follow \citet{kadar2018conll} and 
generate a sentence pair data set
by taking all pairs of sentences 
that belong to the same image
and are written in different languages: 5 English
and 5 German captions result in 25 English-German 
pairs. 
The sentences are encoded and we perform an update of 
the model parameters using the same loss. 
When training with both the image--caption and 
caption--caption (c2c) ranking tasks, 
we randomly select the task to perform with 
probability $p$=0.5. 

\subsection{Generating Synthetic Pairs}\label{sec:method:synthetic}

We propose two approaches to creating synthetic image--caption pairs to improve image--sentence ranking models when training with disjoint data sets. We assume the existence of datasets 
$\mathcal{D}_1$: $\mathcal{I}^1$-$\mathcal{C}^{\ell_1}$ and
$\mathcal{D}_2$: $\mathcal{I}^2$-$\mathcal{C}^{\ell_2}$
consisting of image--caption pairs $<i^1_i, c^{\ell_1}_i>$ and 
$<i^2_i, c^{\ell_2}_i>$ in 
languages $\ell_1$ and $\ell_2$, where the image sets do not overlap 
$\mathcal{I}^1 \cap \mathcal{I}^2 = \varnothing$. 
We seek to extend $\mathcal{I}^2$-$\mathcal{C}^{\ell_2}$ 
to a bilingual dataset with synthetic captions
$\hat{c}^{\ell_1}_i \in \hat{\mathcal{C}}^{\ell_1}$ 
in language $\ell_1$, resulting in a triplet data set 
$\mathcal{I}^2$-$\hat{\mathcal{C}}^{\ell_1}$-$\mathcal{C}^{\ell_2}$ 
consisting of triplets
 $<i^2_i, \hat{c}^{\ell_1}_i, c^{\ell_2}_i>$.
 We hypothesize that the new dataset will improve model performance because it will be trained to map the images to captions in both languages.%, \todo{because training with overlapping images is easier.}
 
 %we can train with the image--sentence and caption--caption ranking objectives \cite{D17-1303,kadar2018conll}.
%Figure \ref{fig:overview} presents an overview of our approaches to creating $\mathcal{I}$-$\mathcal{C}^\hat{\ell_1}$ pairs.

\subsection{Pseudopairs approach}\label{sec:method:pseudo}

\begin{comment}
Given two corpora $\mathcal{D}_1$ and $\mathcal{D}_2$, where $\mathcal{D}_1$ contains $<i_1, c_1>$ and $<i_2, c_2>$ pairs, and $\mathcal{D}_2$ contains $<i_2, c_1>$ pairs, we generate a \emph{pseudo-pair} corpora with noisy image-sentence pairs extracted from $D_1$. %Figure \ref{fig:overview} (b) shows an overview of this process.
In this paper, we create pseudo-pairs
in one direction $\mathcal{D}_1 \rightarrow \mathcal{D}_2$ leading 
to new image--caption pairs $<i_2, \hat{c}_2>$ for $\mathcal{D}_2$. 
We generate these pseudo-pairs using the cosine similarity between
sentence embeddings using a model trained on the $\mathcal{D}_1$ data. 
When generating 
pseudo-pairs $\mathcal{D}_1 \rightarrow \mathcal{D}_2$, we encode all 
captions $c_1 \in \mathcal{D}_2$ using the model trained on $\mathcal{D}_1$, and transfer the most similar $\hat{c}_2$ caption 
from $\mathcal{D}_1$ as its pair. This leads to $<c_2, c_1>$ pairs, and as a consequence to $<i_2, \hat{c}_2>$ pairs, and ultimately in a $\mathcal{D}_2$ corpus that contains $<i_1, c_1>$ and $<i_2, \hat{c_2}>$ pairs.
\end{comment}

Given two image-caption corpora  $\mathcal{I}^1$-$\mathcal{C}^{\ell_1}$  and  $\mathcal{I}^2$-$\mathcal{C}^{\ell_2}$ with pairs 
$<i^1_i, c^{\ell_1}_i>$ and $<i^2_i, c^{\ell_2}_i>$, we generate a \emph{pseudopair} corpus labeling each image in $\mathcal{I}^2$ with 
a caption from $\mathcal{C}^{\ell_1}$.
%Figure \ref{fig:overview} (b) shows an overview of this process.
In our experiments, we create pseudopairs
in only in one direction  leading 
to new image--caption pairs $<i^2, \hat{c}^{\ell_1}>$. 

The pseudopairs are generated using the sentence representations of
the model trained on both corpora  $\mathcal{I}^1$-$\mathcal{C}^{\ell_1}$
and $\mathcal{I}^2$-$\mathcal{C}^{\ell_2}$ jointly. 
We encode all 
captions $c^{\ell_1}_i \in \mathcal{C}^{\ell_1}$ and 
 $c^{\ell_2}_i \in \mathcal{C}^{\ell_2}$ and for each $c^{\ell_2}_i$
find the most similar caption $\hat{c}^{\ell_1}_i$ using the cosine similarity between the sentence representations.
This leads to pairs $<c^{\ell_2}_i, \hat{c}^{\ell_1}_i>$ 
and as a result to triplets $<i^2_i, c^{\ell_2}_i, \hat{c}^{\ell_1}_i>$ .



\begin{comment}


This is summarized in Algorithm~\ref{alg:sentsim}.

\begin{algorithm}
\begin{algorithmic}
\State $pseudo\_caps \Leftarrow [\;]$
\For{$c_2 \in D_2$} 
    \State $\hat{c_2}$ = arg max $sim(c_1, C_1)$
    \State $pseudo\_caps$.add($\hat{c_2}$)
\EndFor
\end{algorithmic}
\caption{Pseudo-pairs based on sentence-similarities.}
\label{alg:sentsim}
\end{algorithm}
\end{comment}

\begin{comment}

\paragraph{2. Image-similarities} are the second option we explored. 
This approach is the same as the sentence similarities just the other way 
around as in as shown in Algorithm~\ref{alg:imgsim}.
\begin{algorithm}
\begin{algorithmic}
\State $pseudo\_imgs \Leftarrow [\;]$
\For{$i_2 \in D_2$} 
    \State $\hat{i_1}$ = arg max $sim(i_2, I_1)$
    \State $pseudo\_imgs$.add($\hat{i_1}$)
\EndFor
\end{algorithmic}
\caption{Pseudo-pairs based on image-similarities.}
\label{alg:imgsim}
\end{algorithm}
\end{comment}

\paragraph{Filtering}

Optionally we filter 
the resulting pseudopair set $\mathcal{C}^{\ell_1}$, in an attempt to
avoid misleading samples with three filtering strategies:

\begin{enumerate}
    \item No filtering.
    \item Keep top: keep items with similarity scores in the 75\% percentile; keep top 25\% 
    \item Remove bottom: keep items with similarity scores in the 25\%; remove bottom 25\%
\end{enumerate}

\paragraph{Fine-tuning vs. restart}
After the pseudopairs are generated we consider two 
options: re-train the model from scratch with all previous 
data sets adding the generated pseudopairs or fine-tunening with
same data sets and the additional pseudopairs.


\subsection{Translation approach}\label{sec:method:mt}
Given a corpus  
$\mathcal{I}^2$-$\mathcal{C}^{\ell_2}$ with pairs 
$<i^2_i, c^{\ell_2}_i>$, we use a machine 
translation system 
to and translate each caption $c^{\ell_2}_i$ to a language $\ell_1$ leading to new image--caption pairs 
$<i^2_i, \hat{c}^{\ell_1}_i>$.
\footnote{\citet{li2016adding} used a similar approach to create Chinese captions for images in the Flickr8K dataset, but they used the translations to train a Chinese image captioning model.} 
Any off-the-shelf translation system could be used to create the translated captions, e.g. an online service, such as Google Translate, or a pre-trained translation model. Here, we use a pre-trained model as it facilitates reproduction. 
%Figure \ref{fig:overview} (a) presents an overview of this approach.

%OpenNMT English-German model \cite{2017opennmt}\footnote{\url{https://s3.amazonaws.com/opennmt-models/wmt-ende_l2-h1024-bpe32k_release.tar.gz}}. The advantage of using the pre-trained system is th


\begin{comment}

Given two corpora $D^{\ell_1}$ and $D^{\ell_2}$ in two languages $\ell_1$ and $\ell_2$
containing image-sentence pairs $<i^{\ell_1}, s^{\ell_1}>$ and  $<i^{\ell_2}, s^{\ell_2}>$,
we generate a new \emph{pseudo-pair} corpus with pseudo image-sentence pairs extracted from the original $D^{\ell_1}$, $D^{\ell_2}$ corpora. Our solution is based on the
intuition that similar images are likely to have similar descriptions or conversely, similar sentences belong to similar images. This can be 
implemented by computing pairwise similarities between the image representations in the
corpora and taking the most similar pairs as new data points in the training data. 
In other words, for each image $i^{\ell_1}$ in $D^{\ell_1}$, we calculate its similarity to 
all of the images in $D^{\ell_2}$, and create a new data point $<i^{\ell_1}, i^{\ell_2}>$. 
From this new data point, we can extrapolate a 4-tuple of the similar images and their 
captions: $<i^{\ell_1}, i^{\ell_2}, s^{\ell_1}, s^{\ell_2}>$. 
The outcome of this process is three new noisy corpora: (1) $<i^{\ell_1}, s^{\ell_2}>$: 
images from $D^{\ell_1}$ paired with captions from $D^{\ell_2}$. (2) 
$<i^{\ell_2}, s^{\ell_1}>$: images from $D^{\ell_2}$ paired with captions 
from $D^{\ell_1}$; and (3) $<s^{\ell_1}, s^{\ell_2}>$: captions from $D^{\ell_1}$ paired 
with captions from $D^{\ell_2}$. We also perform this process in the other direction using the
intuition that similar captions are likely to belong to the similar images. For each caption
we calculate the similarity to all other captions in the other languages and perform the aforementioned procedure from this direction too.

\end{comment}
\section{Experimental Protocol}

\subsection{Model}

Our implementation, training protocol and parameter settings are based on the existing codebase of \cite{kadar2018conll}.
%\footnote{\url{https://github.com/kadarakos/mulisera}}.
In all experiments, we use the 2048 dimensional image-features extracted from the
last average-pooling layer of a pre-trained\footnote{Trained on the ILSVRC 2012 1.2M image 1000 class object classification
subset ImageNet \citep{russakovsky2015imagenet}} 
ResNet50 CNN \citep{he2016deep}. 
  
The image representation used in our model is obtained by a single 
affine transformation that we train from scratch 
$\mathbf{W}_{I} \in \mathbb{R}^{2048 \times 1024}$. For the sentence
encoder we use a uni-directional Gated Recurrent Unit (GRU) 
network \citep{cho2014properties} with a single hidden layer with 
1024  hidden units and 300 dimensional word embeddings. 
When training bilingual models we use a single word-embedding
for the same word-forms, making no distinction if they come 
from different languages. Each sentence is represented by the final 
hidden state of the GRU. For the similarity function in the loss function (Eq. 1) we use cosine similarity and $\alpha=0.2$ margin parameter.

In all experiments we inspect the model performance on the validation 
set at every 500 updates
and stop training when no improvement is observed for 10 inspections. 
The performance metric we use as the stopping criterion is the 
sum of text-to-image (T$\rightarrow$I) and 
image-to-text (I$\rightarrow$T) recall at 1, 5 and 10 
scores across all languages in the training data. 
In all experiments we use a batch-size of 128.
The models are trained with the Adam optimizer \citep{kingma2014adam} 
using default parameters and an initial learning rate of
\mbox{2e-4}
without applying any learning-rate decay schedule.
We apply gradient norm clipping with a value of 2.0.

We use a pre-trained
%\footnote{As opposed to an online service, such as Google Translate to facilitate reproducibility.}
OpenNMT \citep{2017opennmt} English-German model \footnote{\footnotesize{\url{https://s3.amazonaws.com/opennmt-models/wmt-ende_l2-h1024-bpe32k_release.tar.gz}}} to create the data for translation approach described in Section \ref{sec:method:mt}.

\subsection{Datasets}
%\todo{Consider making a Table from the data set statistics.}

The models are trained and evaluated on the bilingual English-German Multi30K dataset (M30K), and we
optionally train on the English COCO dataset 
\citep{Chen2015}. In \textit{monolingual} 
experiments, the model is trained on a single language from M30K or 
COCO.

In the \emph{aligned} bilingual experiments, we use the
independently collected English and German captions in M30K:
%30K images with 5 English captions for each language per image
%(taken from the Flickr30K dataset \cite{young2014image}). 
The training set consists of 29K images and 145K captions; the validation and test sets have 1K images and 5K captions. 

For the \emph{disjoint} experiments, we use the COCO data set
with the \citep{karpathy2015deep} splits. This gives 82,783 training,
5,000 validation, and 5,000 test images; each image is paired with five captions.
The data set has an additional split containing the  
30,504 images from the original validation set of 
MS-COCO (``restval''),
%\footnote{Usually referred to 
% as "restval" meaning the rest of the original validation set .}, 
 which we add to the training set as in previous work \citep{karpathy2015deep,vendrov2016order,faghri2017vse++}.
 
%In all of our experiments we report results and perform early-stopping on the M30K data. We report results on Mutimodal Transaltion Shared Task 2016 Test split \cite{specia2016shared}. We do not use the test and validation sets of COCO in any of our experiments.


\subsection{Evaluation}
We report results on Multimodal Translation Shared Task 2016 test split of M30K \citep{Specia2016}. 
Due to space constraints, we only report recall at 1 (R@1) for 
Image-to-Text (I$\rightarrow$T) and Text-to-Image (T$\rightarrow$I)
retrieval, and the sum of R@1, R@5, and R@10 recall scores across both tasks and languages (Sum).\footnote{This is the criterion we use
for early-stopping.}
%The complete set of results including Recall@5 and Recall@10 can be found in the Supplementary Material.

%Recall@1, 5, and 10 for image-to-text retrieval (T $\rightarrow$ I) and text-to-image retrieval (I $\rightarrow$ T). 

\section{Baseline Results}\label{sec:results}

The experiments presented here set the baseline performance 
for the visually grounded bilingual models
and introduces the data settings that we will use in the 
later sections.

\paragraph{Aligned}
%\todo[inline]{Results of training the model on only the M30K}

\begin{table}[t]
    \centering
    \renewcommand{\arraystretch}{1.0}
    \begin{tabular}{lccc}
        \toprule
         & \multicolumn{3}{c}{English}\\
         \cmidrule(lr){2-4}
         & I $\rightarrow$ T & T $\rightarrow$ I & Sum\\
         \midrule
         En & 40.5 & 28.8 & 346.4 \\
         \; + De & 41.4 & 29.9  & 352.8\\
         \; \; + c2c & 42.8  & 32.1 & 361.6\\
         \hdashline
         COCO & 34.4 & 24.8 & 304.0 \\
         \; + En & \textbf{46.2} & \textbf{33.4} & \textbf{374.4}\\
         \bottomrule
    \end{tabular}
    %\caption{Recall @ 1 and Sum-of-Recall-Scores for Image-to-Text (I$\rightarrow$T) and 
    %Text-to-Image (T$\rightarrow$I) on the English
    %M30K 2016 test set in the \emph{aligned setting}: 
    %only the M30K English (En), M30K German and English (+De), 
    %and with the additional caption ranking objective (+c2c).}
    \caption{Performance on the English
    M30K 2016 test set in the \emph{aligned setting} for models trained on 
    M30K English (En), both M30K German and English 
    (+De), with caption ranking (+c2c), 
    COCO (COCO) and both COCO and M30K English (+En).}
    \label{tab:engbaseline}
\end{table}



\begin{table}
    \centering
    \renewcommand{\arraystretch}{1.0}   
    \begin{tabular}{lccc}
        \toprule
         & \multicolumn{3}{c}{German} \\
         \cmidrule(lr){2-4}
         & I $\rightarrow$ T & T $\rightarrow$ I & Sum\\
         \midrule
         De & 34.9 & 24.6 & 311.2 \\
         \; + En & \textbf{38.6} & 26.0 & 324.6\\
         \; \; + c2c & 38.3 & \textbf{27.7} &   \textbf{334.0} \\
         \; + COCO & 36.4 & 25.7 & 319.7\\
         \bottomrule
    \end{tabular}
        \caption{Performance on the German
    M30K 2016 test set in the \emph{aligned} and
    \emph{disjoint} settings for models trained on 
    M30K German (De), both M30K German and English (+En), 
    and with caption ranking (+c2c) and both M30K German and COCO (+COCO).}
    \label{tab:gerbaseline}
\end{table}

\begin{table*}[h!]
    \centering
    \renewcommand{\arraystretch}{1.0}
    \begin{tabular}{lcccccc}
        \toprule
         & \multicolumn{3}{c}{English} & \multicolumn{3}{c}{German}\\
        \cmidrule(lr){2-4} \cmidrule(lr){5-7}
         & I $\rightarrow$ T & T $\rightarrow$ I & Sum & I $\rightarrow$ T & T $\rightarrow$ I & Sum \\
         \midrule
         En+De+c2c & 42.8 & 28.6 & 361.6 & 38.3 & 27.7 & 334.0 \\
         \; + COCO & \textbf{46.5} & \textbf{34.8} &  \textbf{378.9}  & \textbf{40.6} & \textbf{28.8} & \textbf{344.6}\\
         \bottomrule
    \end{tabular}
    \caption{Recall @ 1 and Sum-of-Recall-Scores for Image-to-Text (I $\rightarrow$ T) and Text-to-Image (T $\rightarrow$ I) baseline results on the English and German
    M30K 2016 test in the \emph{aligned plus disjoint} setting}\label{tab:alignedplus}
\end{table*}

In these experiments we only use the \emph{aligned} English-German data from M30K. 
%This is the setting was studied before in \newcite{D17-1303} and \newcite{kadar2018conll}.
Tables~\ref{tab:engbaseline} and \ref{tab:gerbaseline} present the result for English and German, respectively. The Sum-of-recall scores for both languages show that the best approach is the bilingual model with the c2c loss (En+De+c2c, and De+En+c2c). These results reproduce the findings of
\cite{kadar2018conll}.


\paragraph{Disjoint}\label{sec:baseline_results:disjoint}
%\todo[inline]{Results of training the model on only the M30K}

\begin{comment}
\begin{table}
    \centering
    \renewcommand{\arraystretch}{1.0}
    \begin{tabular}{lccc}
        \toprule
         & \multicolumn{3}{c}{English}\\
         & I $\rightarrow$ T & T $\rightarrow$ I & Sum\\
         \midrule
          COCO & 34.4 & 24.8 & 304.0 \\
         En & 40.5 & 28.8 & 346.4 \\
         \; + COCO & \textbf{46.2} & \textbf{33.4} & \textbf{374.4}\\
         \bottomrule
    \end{tabular}
     \caption{Measuring \emph{domain shift} by testing models trained on
     COCO (COCO), M30K English (En) and both (+COCO) on
     the English M30K 2016 test set.}
    \label{tab:engdisjoint}
\end{table}
\end{comment}

%\begin{table}[t]
%    \centering
%    \renewcommand{\arraystretch}{1.0}
%    \begin{tabular}{lccc}
%        \toprule
%         & \multicolumn{3}{c}{German}\\
%         & I $\rightarrow$ T & T $\rightarrow$ I & Sum\\
%         \midrule
%         De & 34.9 & 24.6 & 311.2 \\
%         \; + COCO & \textbf{36.4} & \textbf{25.7} & \textbf{319.7}\\
%         \bottomrule
%    \end{tabular}
%    \caption{Improving the performance of the German (De) model trained on the M30K German set in the \emph{disjoint} setting by adding COCO (+COCO). Results reported on the German M30K 2016 test set.}\label{tab:gerdisjoint}
%\end{table}


We now determine the performance of the model when it is trained on data drawn from different data sets with no overlapping images.

First we train two English \emph{monolingual}
models: one on the M30K English dataset and one on the English COCO dataset. Both models are evaluated on image--sentence ranking performance on the M30K English test 2016 set. The results in Table~\ref{tab:engbaseline} show that there is a substantial difference in performance in both text-to-image and image-to-text retrieval, depending on whether the model is trained on the M30K or the COCO dataset.
The final row of Table~\ref{tab:engbaseline} shows, however, 
that jointly training on both data sets improves over 
only using the M30K English training data.

We also conduct experiments in the \emph{bilingual disjoint} setting, 
where we study whether it is possible to improve the performance of a German model using the
out-of-domain English COCO data. Table~\ref{tab:gerbaseline}
shows that there is an increase in performance when the model is trained on the disjoint sets, as opposed to
only the in-domain M30K German (compare De against De+COCO). This result is not too surprising as we have observed both the advantage of joint training on both languages in the \emph{aligned} setting and the overlap between the different datasets. 

Finally, we compare the performance of a German model trained in the \emph{aligned}
and \emph{disjoint} settings. We find that a model trained in the \emph{aligned}
setting (De+En) is better than a model trained in the \emph{disjoint} setting (De+COCO), as shown in Table \ref{tab:gerbaseline}.
This finding contradicts the conclusion of \cite{kadar2018conll}, who claimed that the
\emph{aligned} and \emph{disjoint} conditions lead to comparable performance. This is most likely because the disjoint setting in \cite{kadar2018conll} is artificial, in the sense that they used different 50\% subsets of M30K. In our experiments the disjoint image--caption sets are real, in the sense that we trained the models on the two different datasets.
 
%Comparing line 1. and
%line 2, in Table~\ref{tab:engbaseline} we observe that training on COCO results in much lower
%performance then training on M30K, showing a considerable difference in domains. However, 
%line 3. shows that there is a considerable overlap too and jointly training on COCO and M30K 
%results in much higher performance than only training on English M30K. This result is reflected
%in line 2. in Table~\ref{tab:gerdisjoint}, where we see that training jointly on COCO and 
%the German M30K improves over the German M30K monolingual model. However, comparing the 
%results between line 2. in Table~\ref{tab:gerbaseline} and line 2. in Table~\ref{tab:gerdisjoint}
%shows that training on \emph{aligned} data is much more beneficial than on \emph{disjoint}.

\paragraph{Aligned plus disjoint}
\begin{comment}

\begin{table}[t]
    \centering
    \renewcommand{\arraystretch}{1.0}
    \begin{tabular}{lccc}
        \toprule
         & \multicolumn{3}{c}{English}\\
         & I $\rightarrow$ T & T $\rightarrow$ I & Sum\\
         \midrule
         En+De+c2c & 42.8 & 28.6 & 361.6 \\
         \; + COCO & \textbf{46.5} & \textbf{34.8} &  \textbf{378.9}\\
         \bottomrule
    \end{tabular}
    \caption{Recall @ 1 and Sum-of-Recall-Scores for Image-to-Text (I$\rightarrow$T) and 
    Text-to-Image (T$\rightarrow$I) baseline results on the English
    M30K 2016 test set for models in the \emph{aligned plus disjoint} setting: 
    M30K German-English model with c2c loss  ( En+De+c2c) and the same model with CO (+COCO)}\label{tab:engalignedplus}
\end{table}


\begin{table}[ht]
    \centering
    \renewcommand{\arraystretch}{1.0}
    \begin{tabular}{lccc}
        \toprule
         & \multicolumn{3}{c}{English}\\
         & I $\rightarrow$ T & T $\rightarrow$ I & Sum\\
         \midrule
        En+De+c2c & 38.3 & 27.7 & 334.0 \\
         \; + COCO & \textbf{40.6} & \textbf{28.8} & \textbf{344.6} \\
             \bottomrule
    \end{tabular}
    \caption{Recall @ 1 Image-to-Text and Text-to-Image baseline results  on the German
    M30K 2016 test set.}\label{tab:geralignedplus}
\end{table}
\end{comment}


Our final baseline experiments explore the combination of 
\emph{disjoint} and \emph{aligned} data settings. We train an English-German bilingual model with the c2c
objective on M30K, and we also train on the English COCO data. Table~\ref{tab:alignedplus} shows that adding the 
\emph{disjoint} data improves performance for
both English and German compared to training 
solely the \emph{aligned} model. 

\paragraph{Summary}
First we reproduced the findings of \cite{kadar2018conll} 
showing that bilingual joint
training improves over monolingual and using c2c loss
further improves performance. 
Furthermore, we have found that adding the COCO as 
additional training data both when only training on German, and training on both German-English from 
M30K improves performance 
even if the model is trained on data drawn from a different dataset.




%\subsection{Domain shift}

%\todo[inline]{Results of training of model on only COCO and evaluating on M30K.}
%\input{results_disjoint}
%\input{results_joint-synthetic}
\input{chapters/EMNLP/pseudo}
\input{chapters/EMNLP/translation}
\section{Discussion}

\input{chapters/EMNLP/discussion-sentsim.tex}

\input{chapters/EMNLP/discussion-pseudoanalysis.tex}

\begin{comment}
\subsection{Pseudopairs or translations?}

The results in Section~\ref{sec:pseudo} show that our pseudopair 
method consistently improves performance compared to training on disjoint data, however, it is vastly
outperformed by the translation approach. In the \emph{disjoint}
setting with the synthetic translated data, the model achieves a 335.5 Sum-of-Recall-scores, while with pseudopair data, the best performance is only 322.9. The difference is similarly
large in the \emph{aligned plus disjoint} case, with the best result of 742.4 Sum-of-Recall-scores for the translated data, and only 728.2 for the pseudopair method.
\todo[inline]{We should try to draw some conclusions from these results. What does it mean for practical applications in the realistic scenario? Is the power of the pseudopair method limited by the size of the disjoint dastasets?}
\end{comment}
%\section{Bonus Results}

\subsection{Sentence-similarity quality}

\begin{comment}

\begin{table}[]
    \centering
    \begin{tabular}{ccc}
        \toprule
        & EN $\rightarrow$ DE & DE $\rightarrow$ EN \\
        \midrule
        BCN & .579 & .570 \\
        IMG\_PIVOT & .772 & .763 \\
        DPCCA & .826 & .791 \\
        \midrule 
        \emph{align} & .813 & .833 \\
        \emph{align} + c2c & .920 & .906\\
        \bottomrule
    \end{tabular}
    \caption{Results on translation retrieval on the Multi30K translation portion test set.}
    \label{tab:translate}
\end{table}
\end{comment}

\begin{comment}
\begin{table*}[]
    \centering
    \begin{tabular}{llll|lll}
    \toprule
 & & & & EN $\rightarrow$ DE & DE $\rightarrow$ EN \\
 \midrule
         BCN & & & & 57.9 & 57.0 \\
          IMG\_PIVOT & & & &  77.2 & 76.3 \\
         DPCCA & & & &  82.6 & 79.1 \\
    \midrule
        C2C  &En & De & COCO &     \\
        \midrule
        & \checkmark &    \checkmark &   & 81.7   & 80.4  \\
        & \checkmark &    \checkmark & \checkmark  & 82.5    & 81.0    \\
        &  & \checkmark     &   \checkmark &   73.4 &  70.7  \\

        \checkmark    & \checkmark &    \checkmark &   & 90.6   & 91.2  \\
        \checkmark & \checkmark &    \checkmark & \checkmark  & 90.0   & 90.1  \\
        \bottomrule
    \end{tabular}
    \caption{Results on translation retrieval on the Multi30K translation portion validation set. ONLY ONE SEED: 57493}
    \label{tab:translate}
\end{table*}
\end{comment}

\begin{table}[]
    \centering
    \begin{tabular}{lcc}
    \toprule
    & EN $\rightarrow$ DE & DE $\rightarrow$ EN \\
    \midrule
    \citet{rajendran2015bridge} & 57.9 & 57.0 \\
    \citet{D17-1303} &  77.2 & 76.3 \\
    \citet{rotman2018bridging} &  82.6 & 79.1 \\
    \midrule
    En + De & 82.7   & 83.7  \\
    \; + c2c & 91.7   & 92.3  \\
    En + De + COCO & 82.9    & 84.8    \\
    \; + c2c & 91.3   & 91.9  \\
    De + COCO & 71.8 &  76.4  \\
        \bottomrule
    \end{tabular}
    \caption{Results on translation retrieval on the Multi30K translation portion validation set. ONLY ONE SEED: 57493}
    \label{tab:translate}
\end{table}

Here we estimate the capability of our model in identifying
translation equivalence by utilizing the English-German
translation pairs from M30K. 
On the one hand we are interested in how well
do the representations encode translation equivalence when 
no translation data was provided and the model solely learns the 
relationship between languages through using the images 
as a bridge between languages. 
Furthermore, we wish to gauge how much these results 
are improved when using
the c2c loss even if this objective only takes pairs
of captions in different languages belonging to the same image
which is still a noisy supervision for this task.
These results also help us estimate how much 
trust should be attributed to the different models
to generate their own pseudopairs.

In Table~\ref{tab:translate}
we report the R@1 of retrieving the correct translation for 
English sentences given the German caption and vica-versa on 
the translation portion of the test set of M30K.
We compare with the best approaches to our knowledge as
reported by \cite{rotman2018bridging}. 
The best best version of their
model (DPCCA) is a deep partial canonical correlation 
analysis method maximizing the correlation between
captions of the same image conditioned 
on image representations as a third view. 
The Bridge Correlation Network (BCN) 
\cite{rajendran2015bridge} is also trained to maximize the 
correlation between the sentences using images as pivots.
IMG\_PIVOT \cite{D17-1303} is the most comparable architecture
with the only difference that it uses the VGG-19 
features instead of ResNet, sum-of-hinges instead 
of max-of-hinges loss and no c2c objective. 

Our models improve upon the state-of-the-art currently
held by DPCCA. We find this results interesting as 
\cite{rajendran2015bridge} note that DPCCA improves over the
CNN+RNN style models even though it is less complex. However,
in our setup the two models have similar performances and the
model with the c2c loss vastly outperforms DPCCA. 

We find that adding more monolingual English data
in from COCO to the bilingual M30K model improves retrieval
performance. Comparing the aligned bilingual M30K model to the disjoint M30K German and 
English COCO setup we find that disjoint model has a much lower retrieval performance on
this task. This result suggests that the disjoint model generates lower quality pseudo-pairs,
which provides explanation for the worse performance. On the other hand we find a large  
improvement when using the caption--caption loss, which suggests higher quality pseudo-pair 
data for the c2c model, explaining the superior performance.

\subsection{Sum- vs. Max-violation}

%\textbf{Is it better to train an image--sentence model with sum- or max-violation? Faghri does not comprehensively show this because their single-crop results on VGG19 suggest no difference. Most of their best results are with random cropping on ResNet.}

Following \cite{kadar2018conll} in all our experiments we used the max-violation 
objective function -- see Equation~\ref{eq:maxviol} -- , however many architectures apply
the sum-violation -- see Equation~\ref{eq:sumviol} --  in the context of image-sentence 
ranking \cite{nam2017dual} multilingual grounded learning \cite{D17-1303} and 
grounded learning from speech signal \cite{chrupala2017representations}. The article
introducing the max-violation loss function \cite{faghri2017vse++} compares sum- and 
max-violation objectives and do not show improved performance on Flickr30K 
in the setting most similar to
ours: fixed center-crop VGG-19 image features throughout training (as opposed to fine-tuning 
and/or random crops). Here we systematically compare the two loss functions 
in our center-crop 
ResNet50 feature setup in monolingual, bilingual experiments and with the addtional 
caption--caption loss.  The results are shown on Figure~\ref{fig:summax}: each 
value we report is the sum of recall scores R@1, R@5 and R@10
across both T $\rightarrow$ I and I $\rightarrow$ T and both
English and German.
In all conditions we find improvements with the max-violation objective
compared to the sum-violation.

\begin{equation}
\label{eq:sumviol}
\begin{split}
\mathcal{J}(a, b) = \sum_{<\hat{a}, b>}[\text{max}(0, \alpha - s(a,b) + s(\hat{a}, b))] \;+ \\ \sum_{<a, \hat{b}>}[\text{max}(0, \alpha - s(a,b) + s(a, \hat{b}))]
\end{split}
\end{equation}




\begin{figure}
    \centering
    \includegraphics[scale=0.45]{assets/summax.png}
    \caption{Comparing Sum- and Max-violation loss. The reported values are the sum of recall scores 
            across R@1, R@5 and R@10 for both image-to-text and text-to-image retrieval for both English and german.
            Each value is the average of 3 runs.}
    \label{fig:summax}
\end{figure}
\section{Related Work}

%\subsection{Image--sentence ranking}

Image--sentence ranking is the task of retrieving the sentences that best describe an image, and vice-versa \cite{hodosh2013framing}. Most recent approaches are based learning to project image representations and sentence representations into a shared space using deep neural networks \cite[\textit{inter-alia}]{frome2013devise,socher2014grounded,vendrov2016order,faghri2017vse++}. 
%State-of-the-art approaches are now based on attention mechanisms that operate between both inputs \cite{Huang2017InstanceAwareIA,Lee_2018_ECCV}, but there is a continued interest in simpler approaches using average-pooled representations of both inputs \cite{vendrov2016order,faghri2017vse++}. 

More recently, there has been a focus on solving this task using multilingual data \cite{D17-1303,kadar2018conll} in the Multi30K dataset \cite{W16-3210}; an extension of the popular Flickr30K dataset into German, French, and Czech.
These works take a multi-view learning perspective in which 
images and their descriptions in multiple languages are different views of the
same concepts. The assumption is that common representations of multiple languages and perceptual stimuli 
can potentially exploit complementary information
between views to learn better representations. 
For example, \cite{rotman2018bridging} improves bilingual sentence representations by incorporating image information as a third view by Deep Partial Canonical Correlation Analysis.
More similar to our work 
\cite{D17-1303}, propose a convolutional-recurrent architecture with both an image--caption
and caption--caption loss to learn bilingual visually grounded representations. 
Their results were improved by the approach presented in \cite{kadar2018conll}, who also
shown the multilingual models outperform bilingual models, and that image--caption retrieval 
performance in languages with less resources can be improved with data from higher-resource
languages. We largely follow \citet{kadar2018conll}, however, our main interest lies in learning multimodal
and bilingual representations in the scenario where the images do not come from the same
data set i.e.: the data is presented is two sets of image--caption tuples rather than
image--caption--caption triples.

Taking a broader perspective, images have been used as pivots in multilingual multimodal language processing. On the word level this intuition is 
applied to visually grounded bilingual lexicon induction, which aims to learn 
cross-lingual word representations without aligned text using images as pivots
\cite{bergsma2011learning,kiela2015visual,vulic2016multi,hartmann2017limitations,hewitt2018learning}. Images have been used as pivots to learn translation models only from image--caption
data sets, without parallel text \cite{hitschler2016multimodal,nakayama2017zero,lee2017emergent,chen2018zero}.

%In this paper, we study approaches to training better image--sentence ranking models using out-of-domain datasets that do not have annotations in all of the necessary languages.

%Perceptual grounding in general and the visual modality specifically 
%can act as a  

%Rather than considering images as pivots between languages one can take the perspective
%that images and their descriptions in multiple languages are different views of the
%same concepts. Taking the this perspective of multi-view learning, 
%common representations of multiple languages and perceptual stimuli 
%can potentially exploit the complementary information
%between views to learn better representations. 

\section{Conclusions}

Previous work has demonstrated improved image--sentence ranking performance
when training models jointly on multiple languages \citep{gella2017image, kadar2018conll}. In this paper, we presented a study where we learn multimodal
and multilingual representations in the \emph{disjoint setting}, where
images between languages do not overlap.
We found that learning visually grounded sentence embeddings in this
setting is more challenging then when the images are \emph{aligned}
between languages. To close the gap, we developed a \emph{pseudopairing} technique  that creates synthetic pairs by annotating the images
of one of the data set with the image--descriptions of the other using the
sentence similarities of the model trained on both. We showed that 
training with the pseudopairs improves the performance of the model
without the need to augment training from additional data sources or 
other pipeline components. However, our technique is outperformed
by creating synthetic pairs using an off-the-shelf automatic machine translation system. As such our results suggest that it is better to use translation,
when a good translation system is available, however, in its absence, pseudopairs
offer consistent improvements. The pseudopairing method only transfers annotations from a small number of images; in the future we plan to substitute our naive matching algorithms
with approaches developed to mitigate this hubness issue \citep{radovanovic2010existence}.%,tomavsev2011influence,tomavsev2011probabilistic,dinu2014improving}.

%\section{Introduction}
%\section{Method}
%\subsection{Synthetic by Translation}
%\subsection{Synthetic by Pseudopairs}
%\section{Experimental Protocol}
%\subsection{Datasets}
%\subsection{Hyperparameters} % I always call this implementation details si think
%\section{Results}
%\subsection{Baselines}
%\subsection{Domain Shift}
%\subsection{Caption-Caption Retrieval} 
% This i was thinking is in the last bit to part with 
%\section{Training with Disjoint Data}
%\section{Joint Training with Synthetic Data}
%\section{Discussion}
%\section{Related Work}
%\section{Conclusion}

%\bibliography{naaclhlt2019}
%\bibliographystyle{acl_natbib}

%\begin{table*}[]
    \centering
    \renewcommand{\arraystretch}{1.0}
    \begin{tabular}{c|cccccc|cccccc}
        \hline
            &&&&&&& \multicolumn{3}{c}{I $\rightarrow$ T} & \multicolumn{3}{c}{T $\rightarrow$ I} \\
         & \rotatebox{90}{De} & \rotatebox{90}{En} & \rotatebox{90}{COCO} & 
         \rotatebox{90}{OpenNMT-De} & \rotatebox{90}{Pseudo-De} & \rotatebox{90}{c2c}  & R@1 & R@5 & R@10 & R@1 & R@5 & R@10 \\
         \hline
         & \multicolumn{11}{l}{\textit{In-domain Multi30K}} \\
         \hline
         1 &  \checkmark  & & & & & & 34.9 & 64.1  & 76.2  & 24.6 & 50.1 & 61.3 \\
         2 & \checkmark & \checkmark & & & &  & \textbf{38.6}  & 66.9  &  77.9 & 26.0 & 52.2  & 63.0\\
         3 & \checkmark & \checkmark &  & & & \checkmark & 38.3 & \textbf{68.3} & \textbf{78.8} & \textbf{27.7} & \textbf{54.9} & \textbf{66.0}  \\
         \hline
         & \multicolumn{11}{l}{\textit{In-domain and out-of-domain COCO}} \\
         \hline
         4 & \checkmark & & \checkmark & &  &  & 36.4 & 65.9  & 77.6 & 25.7  & 51.6 & 62.6 \\
         5 & \checkmark & \checkmark & \checkmark & & & & 39.2  & 67.8  & 79.1 & 27.1 & 53.2  &  64.0  \\
         6 & \checkmark & \checkmark & \checkmark & & & \checkmark  & \textbf{40.6} & \textbf{70.5}  & \textbf{81.1}  & \textbf{28.8} & \textbf{56.3}  & \textbf{67.3} \\
         \hline
         & \multicolumn{11}{l}{\textit{In-domain and distant supervision: translation}} \\
         \hline
         7 & \checkmark & &  & \checkmark & & & 38.5 & 68.4 & 79.6 & 26.6 & 53.7 & 64.4\\
         8 & \checkmark & & \checkmark  & \checkmark & & & 39.1 & 69.8 & 80.2 & 27.5 & 54.2 & 64.7 \\
         9 & \checkmark & \checkmark &  & \checkmark & & & 39.8 & 69.6 & 81.0 & 27.6 & 54.8 & 65.5 \\
         10 & \checkmark & \checkmark &  & \checkmark & & \checkmark & 38.2 & 66.3 & 77.0 & 27.2 & 54.2 & 65.5 \\
         11 & \checkmark & \checkmark & \checkmark  & \checkmark & & & 40.2 & 71.2 & 81.9 & 28.3 & 55.6 & 66.1 \\
         12 & \checkmark & \checkmark & \checkmark  & \checkmark & & \checkmark & \textbf{43.5} & \textbf{73.3} & \textbf{82.5} & \textbf{30.5} & \textbf{58.8} & \textbf{69.3} \\

         \hline
         & \multicolumn{11}{l}{\textit{Distant supervision: psuedopairs}} \\
          \hline
         13 &\checkmark & & \checkmark  &  & \checkmark & & 36.7 &  65.1  & 76.3  & 25.2  & 51.3 &   62.2\\
         13 &\checkmark & & \checkmark  &  & 25\% & & 36.6 &  65.2  & 76.8  & 25.0  & 51.6 &   62.9\\
         13 &\checkmark & & \checkmark  &  & 75\% & & 36.7 &  63.8  & 75.8   & 24.6  & 51.1 &   62.1\\
         \hdashline
         14 & \checkmark & & \checkmark  &  & \checkmark & \checkmark  & 35.6  & 65.3 & 75.6  & 25.0  &  51.2 & 62.4 \\
         14 & \checkmark & & \checkmark  &  & 25\% & \checkmark  & 33.6  & 63.5 & 75.6 & 23.9 &  50.3 & 61.5 \\
         14 & \checkmark & & \checkmark  &  & 75\% & \checkmark  &  &  &  &  &  & \\
         \hdashline
         15 & \checkmark & \checkmark & \checkmark  &  & \checkmark &  & 37.9  &  68.3  & 78.8 &  26.8 & 53.9 & 64.6 \\
        15 & \checkmark & \checkmark & \checkmark  &  & 25\% &  &  38.4 & 68.9 & 78.9 &  26.8 & 53.3 & 64.4  \\
         15 & \checkmark & \checkmark & \checkmark  &  & 75\% &  & 37.7 & 66.8 & 78.7 & 26.2  & 53.1 & 63.6 \\
         \hdashline
         16 & \checkmark & \checkmark & \checkmark  &  & \checkmark & \checkmark  & 39.9 & 69.9  & 80.2 & 29.6 &  57.3 &  67.8 \\
         16 & \checkmark & \checkmark & \checkmark  &  & 25\% & \checkmark  & 41.5 & 70.4 & 80.8  & 28.9  & 56.8  & 67.2\\
         16 & \checkmark & \checkmark & \checkmark  &  & 75\% & \checkmark  & 40.3  &  69.3 &    79.1  &  27.9 & 55.6 & 66.8\\
    \end{tabular}
    \caption{Results reported on German}
    \label{tab:German}
\end{table*}



\begin{table*}[]
    \centering
    \renewcommand{\arraystretch}{1.0}
    \begin{tabular}{c|cccccc|cccccc}
        \hline
            &&&&&& \multicolumn{3}{c}{I $\rightarrow$ T} & \multicolumn{3}{c}{T $\rightarrow$ I} \\
         & \rotatebox{90}{De} & \rotatebox{90}{En} & \rotatebox{90}{COCO} & 
         \rotatebox{90}{OpenNMT-De} & \rotatebox{90}{Pseudo-De} & \rotatebox{90}{c2c}  & R@1 & R@5 & R@10 & R@1 & R@5 & R@10 \\
         \hline
         & \multicolumn{11}{l}{\textit{In-domain Multi30K}} \\
         \hline
         1 &  & \checkmark & & & &  & 40.5  &  69.6  &  79.9   & 28.8 & 58.3 & 69.4  \\
         2 &  &  & \checkmark & & & & 34.4  & 61.4  &  72.1 & 24.8 & 50.3  & 61.1\\
         3 & \checkmark & \checkmark & & & &   & 41.4 & 70.9 & 81.0 & 29.9 & 59.6 & 70.0 \\
         4 & \checkmark & \checkmark &  & & & \checkmark & \textbf{42.8} & \textbf{72.0} & \textbf{81.2}  & \textbf{32.1} & \textbf{61.4} & 
          \textbf{72.2} \\
         \hline
         & \multicolumn{11}{l}{\textit{In-domain + out-of-domain COCO}} \\
         \hline
         5 & & \checkmark & \checkmark & &  &  & 46.2 & 75.6 & 83.6 & 33.4 & 62.5  & 73.1  \\
         6 & \checkmark & \checkmark & \checkmark & & &  & 46.0  & 75.3 & 83.9  &33.6 & 62.7 &  73.3  \\
         7 & \checkmark & \checkmark & \checkmark & & & \checkmark  & \textbf{46.5} & \textbf{75.2} & \textbf{83.8} & \textbf{34.8} & \textbf{64.4}  & \textbf{74.3} \\
         \hline
         & \multicolumn{11}{l}{\textit{Distant supervision: translation}} \\
         \hline
         8 & & \checkmark &  & \checkmark & & & 42.2 & 71.1 & 80.6 & 30.5 & 57.8 & 68.7 \\
         9 & & \checkmark & \checkmark  & \checkmark & & & 45.6 & 72.9 & 81.1 & 33.0 & 61.0 & 71.8  \\
         10 & \checkmark & \checkmark &  & \checkmark & & & 43.8 & 72.0 & 81.4 & 30.9 & 60.1 & 70.9 \\
         11 & \checkmark & \checkmark &  & \checkmark & & \checkmark & 44.9 & 73.2 & 82.1 & 33.8 & 61.7 & 72.2 \\
         12 & \checkmark & \checkmark & \checkmark  & \checkmark & & & 46.8 & 74.7 & 83.4 & 33.6 & 62.9 & 73.2 \\
         13 & \checkmark & \checkmark & \checkmark  & \checkmark & & \checkmark & \textbf{47.5} & \textbf{75.7} & \textbf{84.1} & \textbf{36.2} & \textbf{65.6} & \textbf{75.6} \\

         \hline
         \multicolumn{12}{l}{\textit{Distant supervision: psuedopairs}} \\
         \hline
         14 & \checkmark & \checkmark & \checkmark  &  & \checkmark &  & 46.4  & 75.4  & 83.2  &  33.3 & 62.5 & 73.1\\
         15 & \checkmark & \checkmark & \checkmark  &  & 25\% &  & 46.9  & 74.9  & 83.4   &  33.7 & 62.8 & 73.1\\
        17 & \checkmark & \checkmark & \checkmark  &  & 75\% &  & 46.1  & 75.1 & 83.7  &  33.5  & 62.9  & 73.3 \\
        \hdashline
         18 & \checkmark & \checkmark & \checkmark  &  & \checkmark & \checkmark  &  47.7 & 75.3   &   83.7 & 35.3 & 65.1 & 75.2 \\
        19 & \checkmark & \checkmark & \checkmark  &  & 25\% & \checkmark  & 47.5  & 75.3 & 83.8 & 34.9  & 64.2 &  74.4 \\
        20 & \checkmark & \checkmark & \checkmark  &  & 75\% & \checkmark  & 45.9 & 73.5  & 82.6  & 34.0 & 63.6 & 74.0  \\
    \end{tabular}
    \caption{Results reported on English}
    \label{tab:English}
\end{table*}

