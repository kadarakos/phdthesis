\section{Experimental Protocol}

\subsection{Model}

Our implementation, training protocol and parameter settings are based on the existing codebase of \cite{kadar2018conll}.
%\footnote{\url{https://github.com/kadarakos/mulisera}}.
In all experiments, we use the 2048 dimensional image-features extracted from the
last average-pooling layer of a pre-trained\footnote{Trained on the ILSVRC 2012 1.2M image 1000 class object classification
subset ImageNet \cite{russakovsky2015imagenet}} 
ResNet50 CNN \cite{he2016deep}. 
  
The image representation used in our model is obtained by a single 
affine transformation that we train from scratch 
$\mathbf{W}_{I} \in \mathbb{R}^{2048 \times 1024}$. For the sentence
encoder we use a uni-directional Gated Recurrent Unit (GRU) 
network \cite{cho2014properties} with a single hidden layer with 
1024  hidden units and 300 dimensional word embeddings. 
When training bilingual models we use a single word-embedding
for the same word-forms, making no distinction if they come 
from different languages. Each sentence is represented by the final 
hidden state of the GRU. For the similarity function in the loss function (Eq. 1) we use cosine similarity and $\alpha=0.2$ margin parameter.

In all experiments we inspect the model performance on the validation 
set at every 500 updates
and stop training when no improvement is observed for 10 inspections. 
The performance metric we use as the stopping criterion is the 
sum of text-to-image (T$\rightarrow$I) and 
image-to-text (I$\rightarrow$T) recall at 1, 5 and 10 
scores across all languages in the training data. 
In all experiments we use a batch-size of 128.
The models are trained with the Adam optimizer \cite{kingma2014adam} 
using default parameters and an initial learning rate of
\mbox{2e-4}
without applying any learning-rate decay schedule.
We apply gradient norm clipping with a value of 2.0.

We use a pre-trained
%\footnote{As opposed to an online service, such as Google Translate to facilitate reproducibility.}
OpenNMT \cite{2017opennmt} English-German model \footnote{\footnotesize{\url{https://s3.amazonaws.com/opennmt-models/wmt-ende_l2-h1024-bpe32k_release.tar.gz}}} to create the data for translation approach described in Section \ref{sec:method:mt}.

\subsection{Datasets}
%\todo{Consider making a Table from the data set statistics.}

The models are trained and evaluated on the bilingual English-German Multi30K dataset (M30K), and we
optionally train on the English COCO dataset 
\cite{DBLP:journals/corr/ChenFLVGDZ15}. In \textit{monolingual} 
experiments, the model is trained on a single language from M30K or 
COCO.

In the \emph{aligned} bilingual experiments, we use the
independently collected English and German captions in M30K:
%30K images with 5 English captions for each language per image
%(taken from the Flickr30K dataset \cite{young2014image}). 
The training set consists of 29K images and 145K captions; the validation and test sets have 1K images and 5K captions. 

For the \emph{disjoint} experiments, we use the COCO data set
with the \cite{karpathy2015deep} splits. This gives 82,783 training,
5,000 validation, and 5,000 test images; each image is paired with five captions.
The data set has an additional split containing the  
30,504 images from the original validation set of 
MS-COCO (``restval''),
%\footnote{Usually referred to 
% as "restval" meaning the rest of the original validation set .}, 
 which we add to the training set as in previous work \cite{karpathy2015deep,vendrov2016order,faghri2017vse++}.
 
%In all of our experiments we report results and perform early-stopping on the M30K data. We report results on Mutimodal Transaltion Shared Task 2016 Test split \cite{specia2016shared}. We do not use the test and validation sets of COCO in any of our experiments.


\subsection{Evaluation}
We report results on Multimodal Translation Shared Task 2016 test split \cite{specia2016shared} of M30K. 
Due to space constraints, we only report recall at 1 (R@1) for 
Image-to-Text (I$\rightarrow$T) and Text-to-Image (T$\rightarrow$I)
retrieval, and the sum of R@1, R@5, and R@10 recall scores across both tasks and languages (Sum).\footnote{This is the criterion we use
for early-stopping.}
%The complete set of results including Recall@5 and Recall@10 can be found in the Supplementary Material.

%Recall@1, 5, and 10 for image-to-text retrieval (T $\rightarrow$ I) and text-to-image retrieval (I $\rightarrow$ T). 
