\subsection{Sentence-similarity quality}

\begin{table}[]
    \centering
    \begin{tabular}{lcc}
    \cmidrule[0.08em](r){1-3}
    & EN {\small $\rightarrow$} DE & DE {\small $\rightarrow$} EN \\
    \cmidrule(rl){1-3}
    BCN & 57.9 & 57.0 \\
    IMG\_PIVOT &  77.2 & 76.3 \\
    DPCCA &  82.6 & 79.1 \\
    \cmidrule(rl){1-3}
    En + De & 82.7   & 83.4  \\
    En + De + c2c & \textbf{90.6} & \textbf{91.2}  \\
    En + De + COCO & 82.5    & 81.0    \\
    En + De + COCO + c2c & 90.0   & 90.1  \\
    De + COCO & 73.4 &  70.7  \\
    \cmidrule[0.08em](r){1-3}
    \end{tabular}
    \caption{Translation retrieval results (Recall @ 1) on the M30K 2016 test set compared to the state of the art.}
    \label{tab:translate}
\end{table}

The core of the proposed pseudopairing method is based on measuring the similarity between sentences, but how well does our model encode similar sentences? Here we analyse the ability of our models to identify translation equivalent sentences using the English-German translation pairs in the Multi30K test 2016 data. This experiment proceeds as follows: (i) we assume a pre-trained image--sentence ranking model, (ii) we encode the German and English sentences using the language encoder of the model, (iii) we calculate the model's performance on the task of ranking the correct translation for English sentences, given the German caption, and vice-versa. 
%DPCCA is a deep partial canonical correlation analysis method that maximizes the correlation between captions of the same image conditioned on image representations. The Bridge Correlation Network (BCN) \cite{rajendran2015bridge} is trained to maximize the correlation between the sentences using images as pivots. IMG\_PIVOT \cite{D17-1303} is most similar to our approach, except it uses VGG19 image features instead of ResNet, the sum-of-hinges instead of max-of-hinges loss, and no c2c objective. 

Table~\ref{tab:translate} reports the results of this experiment; we compare against the best approaches as reported by \cite{rotman2018bridging}. Our models consistently improve upon the state-of-the-art. The baseline aligned model trained on the Multi30K data slightly outperforms the DPCCA for EN $\rightarrow$ DE retrieval, and more substantially outperforms DPCAA for DE $\rightarrow$ EN. If we train the same model with the additional caption--caption retrieval objective, R@1 improves by 8.0 and 12.1 points, respectively. We find that adding more monolingual English data from the external COCO data set slightly degrades retrieval performance, and that performing sentence retrieval using a model trained on the disjoint M30K German and English COCO data sets result in much lower retrieval performance. We conclude that the model that we used to estimate sentence similarity is the best-performing method known for this task on this data set, but there is room for improvement for models trained on disjoint data sets.

\begin{comment}
Here we estimate the capability of our model to identify
translation equivalence by utilizing the English-German
translation pairs from M30K. 
Firstly these results help us estimate how much 
trust should be attributed to the different models
to generate their own pseudopairs using 
the sentence-similarity strategy.
Furthermore, we are interested how well
do the representations encode translation equivalence even if 
no translation data was provided i.e. models without 
the c2c loss solely learn the 
relationship between languages through using the images 
as pivots. We then move on to gauge how much these 
results are improved when using
the c2c loss. Note that this objective takes pairs
of captions in different languages belonging to the same image
as such it is still a noisy supervision for this task.

In Table~\ref{tab:translate}
we report the R@1 of retrieving the correct translation for 
English sentences given the German caption and vica-versa on 
the translation portion of the test set of M30K.
We compare with the best approaches to our knowledge as
reported by \cite{rotman2018bridging}. 
DPCCA is a deep partial canonical correlation 
analysis method  maximizing the correlation between
captions of the same image conditioned 
on image representations. The Bridge Correlation Network (BCN) 
\cite{rajendran2015bridge} is also trained to maximize the 
correlation between the sentences, but using images as pivots.
IMG\_PIVOT \cite{gella2017image} is the most comparable architecture
with the only difference that it uses the VGG-19 
features instead of ResNet, sum-of-hinges instead 
of max-of-hinges loss and no c2c objective. 

Our models substantially improve upon the state-of-the-art. We find this results interesting as 
\cite{rajendran2015bridge} note that DPCCA improves over the
CNN+RNN style models even though it is less complex. However,
our setup outperform the DPCCA even without 
explicitly being trained to match sentences.
Adding the c2c loss brings our approach closer to the 
DPCCA in that here we also explicitly push sentences closer 
to each other in the embedding space and we find large
improvements. 

We find that adding more monolingual English data
in from COCO to the bilingual M30K model slightly degrades retrieval performance. Lastly the disjoint De+COCO
results in a much lower retrieval performance on
this task suggesting that this model generates 
lower quality pseudo-pair.
\end{comment}