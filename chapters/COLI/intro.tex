
\section{Introduction}
\label{sec:intro}
Recurrent neural networks (RNNs) were introduced by
\cite{elman1990finding}
as a connectionist architecture with the
ability to model the temporal dimension. They have proved popular for
modeling language data as they learn representations of words and
larger linguistic units directly from the input data, without feature
engineering. Variations of the RNN architecture have been applied in
several NLP domains such as parsing \citep{vinyals2015grammar} and
machine translation \citep{Bahdanau2015}, as well as in computer
vision applications such as image generation \citep{gregor2015draw} and
object segmentation \citep{visin2015reseg}. RNNs are also important
components of systems integrating vision and language, e.g.\ image
\citep{karpathy2015deep} and video captioning \citep{yu2015video}.

These networks can represent variable-length linguistic expressions by
encoding them into a fixed-size low-dimensional vector. The nature and
the role of the components of these representations are not directly
interpretable as they are a complex, non-linear function of the
input. There have recently been numerous efforts to visualize deep
models such as convolutional neural networks in the domain of computer
vision, but much less so for variants of RNNs and for language
processing.

The present paper develops novel methods for uncovering abstract
linguistic knowledge encoded by the distributed representations of RNNs,
with a specific focus on analyzing the hidden activation patterns rather
than word embeddings and on the syntactic generalizations
that models learn to capture. In the current work we apply our methods
to a specific architecture trained on specific tasks, but also provide
pointers about how to generalize the proposed analysis to other settings.

As our case study we picked the {\sc Imaginet} model introduced by \label{explainimaginet}
\cite{chrupala2015learning}. It is a multi-task, multi-modal
architecture consisting of two Gated-Recurrent Unit (GRU)
\citep{cho2014properties,chung2014empirical} pathways and
a shared word embedding matrix. One of the GRUs ({\sc Visual}) is
trained to predict image vectors given image descriptions, while the other
pathway ({\sc Textual}) is a language model, trained to sequentially predict each
word in the descriptions. This particular
architecture allows a comparative analysis of the hidden activation
patterns between networks trained on two different tasks, while
keeping the training data and the word embeddings fixed. Recurrent neural
language models akin to {\sc Textual} which are trained to predict the
next symbol in a sequence are relatively well understood, and there
have been some attempts to analyze their internal states
\citep[among others]{elman1991distributed,karpathy2015visualizing}. In
constrast, {\sc Visual} maps a complete sequence of words to
a representation of a corresponding visual scene and is a less
commonly encountered, but a more interesting model from the point of
view of representing meaning conveyed via linguistic structure.
For comparison, we also consider a standard standalone language model.

We report a thorough quantitative analysis to provide a linguistic
interpretation of the networks' activation patterns. We present a
series of experiments using a novel method we call \emph{omission score}
to measure the importance of input tokens to the final prediction of models
that compute distributed representations of sentences.
Furthermore, we introduce a more global measure for estimating the
informativeness of various types of n-gram contexts for each model.
These techniques can be applied to various RNN architectures, Recursive
Neural Networks and Convolutional Neural Networks.

Our experiments show that the {\sc Visual} pathway in general
pays special attention to syntactic categories which carry semantic
content, and particularly to nouns. More surprisingly, this pathway also
learns to treat word types differently depending on their
grammatical function and their position in the sequential structure of
the sentence. In contrast, the {\sc Textual} pathway and the standalone
language model are especially sensitive to the local syntactic
characteristics of the input sentences.  Further analysis of the most
informative n-gram contexts for each model shows that while the
{\sc Visual} pathway is mostly sensitive to lexical (i.e., token n-gram)
contexts, the language models react more strongly to abstract contexts
(i.e., dependency relation n-grams) that represent syntactic constructions.
